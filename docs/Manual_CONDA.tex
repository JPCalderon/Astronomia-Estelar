\documentclass[a4paper, 12pt]{article}
\usepackage[activeacute, spanish]{babel}
\decimalpoint

\usepackage{hyperref, url}

\RequirePackage{xcolor} % [dvipsnames] 
\definecolor{Maroon}{cmyk}{0, 0.87, 0.68, 0.32}
\definecolor{halfgray}{gray}{0.55}
\definecolor{webbrown}{rgb}{.6,0,0}
\definecolor{webgreen}{rgb}{0,.5,0}
\definecolor{RoyalBlue}{cmyk}{1, 0.50, 0, 0}
\definecolor{byzantium}{rgb}{0.44, 0.16, 0.39}
\definecolor{byzantine}{rgb}{0.74, 0.2, 0.64}
\hypersetup{%
  colorlinks=true, linktocpage=true, pdfstartpage=1, pdfstartview=FitV,%
  urlcolor=byzantium, linkcolor=RoyalBlue, citecolor=byzantine, pagecolor=RoyalBlue,%
  pdftitle={Manual Conda}, pdfauthor={Juan Pablo Calderón}
}

\usepackage{pxfonts}
\usepackage{parskip}
\setlength{\parskip}{1.3ex plus 0.2ex minus 0.2ex} % linea entre parrafos
\setlength{\parindent}{1em} % sangria

\usepackage{fancyvrb}
\usepackage{fvextra}

\DefineVerbatimEnvironment{code}{Verbatim}{fontsize=\small, breaklines=true, commentchar=\#}
\DefineVerbatimEnvironment{example}{Verbatim}{fontsize=\small}
\usepackage{tcolorbox}

\title{Manual CONDA}
\author{Juan Pablo Calderón}
\date{August 2020}

\begin{document}

\maketitle

\section{Introduction}
{\sc Conda}\footnote{\url{https://docs.conda.io/en/latest/}} es un sistema de gestión de paquetes de código abierto que puede ejecutarse en diferentes sistemas operativos (SOs), y forma parte de {\sc Anaconda}\footnote{\url{https://www.anaconda.com/products/individual}}, un conjunto de {\it software} usado principalmente en ciencia.

Está escrito en {\it python}, pero puede gestionar proyectos de otros lenguajes. Permite aislar recursos como librerías y entornos de ejecución del sistema principal o de otros entornos virtuales. Esto significa que en el mismo sistema, computadora, es posible tener instaladas múltiples versiones de una misma librería sin crear ningún tipo de conflicto.

\section{Cómo crear un entorno de ejecución}
En esta sección vamos a explicar cómo se crea un entorno de ejecución ({\it environment}) en un SO donde ya se encuentra instalado {\sc CONDA}, de forma de poder migrar dicho entorno a otro SO y ejecutarlo sin la necesidad de tener instalado {\sc CONDA}.

\begin{tcolorbox}
\begin{code}
$ cd ~/materias/Astronomia_Estelar 
$ conda create -y -p .conda/envs/AEpracticas python=3.7 
$ conda activate .conda/envs/AEpracticas/

(AEpracticas) usuarix1@maquina1:~/materias/Astronomia_Estelar$ 
\end{code}
\end{tcolorbox}

Con lo anterior se crea un entorno aislado del SO original, con la versión de {\it python} $3.7$. Queda identificado entre paréntesis 
(\verb|(AEpracticas)|) el nombre del entorno con el que es identificado dentro de {\sc CONDA}. Notar que para activarlo, en este caso, es necesario indicar el directorio exacto donde fue creado el entorno.

Para remover el entorno:
\begin{tcolorbox}
\begin{code}
(AEpracticas) usuarix1@maquina1:~/materias/Astronomia_Estelar$ 
$ conda env remove -p .conda/envs/AEpracticas

Remove all packages in environment      
     ~/materias/Astronomia_Estelar/.conda/envs/AEpracticas:
\end{code}
\end{tcolorbox}

\subsection{Instalación de paquetes}
Ahora, instalamos dentro del entorno, los paquetes de {\it python} necesarios para la resolución de todas las prácticas.

\begin{tcolorbox}
\begin{code}
(AEpracticas) usuarix1@maquina1:~/materias/Astronomia_Estelar$ 
$ pip install matplotlib
$ pip install pandas
$ pip install specutils
$ pip install lineid_plot
$ pip install specutils
$ pip install lmfit

$ pip install jupyter
$ pip install conda-pack
\end{code}
\end{tcolorbox}


\section{Migrando entornos}
Existen varias formas de migrar un entorno entre SOs: 

\begin{enumerate}
    \item Utilizando\\
    \verb|$ conda list --explicit > spec-list.txt|,\\
    que crea una lista de los paquetes instalados de un entorno, y pueden ser reintalados en otro mediante\\ \verb|$ conda create --name nuevo-entorno --file spec-list.txt|.
    
    \item Utilizando archivos en formato YAML,\\
    \verb|conda env export > environment.yml|,\\ se crea una lista de los paquetes instalados (que incluye los instalados por \verb|pip|). Y mediante\\
    \verb|conda env create -f environment.yml|,\\
    se puede re-crear en otro lugar. Esto permitiría
    migrar entornos entre Linux y Windows, pero la desventaja que tiene es que eventualmente no existan las mismas versiones de los paquetes en uno u otro.
    
    \item Mediante \verb|conda-pack|, que comprime todo el entorno (incluyendo librerías, binarios y paquetes instalados mediante \verb|pip|).
    \end{enumerate}

\subsection{conda-pack}
Para empaquetar el entorno y migrarlo a otra maquina, es necesario ejecutar:
%
\begin{tcolorbox}
\begin{code}
(AEpracticas) usuarix1@maquina1:~/materias/Astronomia_Estelar$ 
$ conda pack -p .conda/envs/AEpracticas
\end{code}
\end{tcolorbox}

Esto generará un archivo (\verb|AEpracticas.tar.gz|) que contiene todo lo necesario para construir el entorno en cualquier directorio.

\subsection{Creando el entorno en otro lado}
Ahora podemos copiar el ambiente empaquetado a otro directorio y
simplemente descomprimirlo y activarlo. 

\begin{tcolorbox}
\begin{code}
usuarix2@maquina2:~/materias/Astronomia_Estelar$
$ mkdir -p .conda/envs/AEpracticas 
$ tar -xzf AEpracticas.tar.gz -C .conda/envs/AEpracticas

$ source .conda/envs/AEpracticas/bin/activate
(AEpracticas) usuarix2@maquina2:~/materias/Astronomia_Estelar$ 

$ conda-unpack
\end{code}
\end{tcolorbox}

De esta forma, se puede tener el mismo entorno en otra maquina, sin necesidad de tener instalado \verb|conda|.

\section{En Windows}
En sistemas operativos Windows, el procedimiento es similar. Cambiaran, obviamente, las rutas de los directorios. Pero además,
la forma de activar el ambiente sería:
\begin{tcolorbox}
\begin{code}
$ .conda\envs\AEpracticas\Scripts\activate.bat
\end{code}
\end{tcolorbox}

Que debe ejecutarse en una consola {\sc DOS}, en el directorio donde ya se ha descomprimido el entorno.
Recordar que en Windows, el comando \verb|ls| se reemplaza por el \verb|dir|, y el \verb|ls -l| por \verb|dir \a|.

\section{Archivos}

\url{http://fcaglp.unlp.edu.ar/~jpcalderon/materias/Astronomia_Estelar/conda/Linux-x86_64/AEpracticas.tar.gz}


\url{http://fcaglp.unlp.edu.ar/~jpcalderon/materias/Astronomia_Estelar/conda/Windows-x86_64/AEpracticas.tar.gz}

\section{Referencias}
Información extraída de la página de \href{https://www.anaconda.com/blog/moving-conda-environments}{Anaconda}.

\end{document}

TODO

1. Agregar sugerencias de Juani
Buenas!

Tengo unos comentarios respecto del manual nomás.
Si el entorno se crea usando nombre (opción n) no se necesita 
especificar el directorio donde va a alojarse y se activa con el nombre.
Yo recomendaría instalar todos los paquetes que se pueda con conda al 
momento de la creación (además ya saben cuales van a ser) para que no 
haya problemas de dependencias (lo que no esté en los repos de conda 
sí instalarlo con pip):
   conda create -y -n AEpracticas python=3.7 matplotlib pandas 
notebook conda-pack
   conda activate AEpracticas
   conda install -c conda-forge specutils
   pip install lmfit (lmfit también está en el canal conda-forge pero 
desactualizado)

Saludos,

Juani Rodriguez
Soporte IALP